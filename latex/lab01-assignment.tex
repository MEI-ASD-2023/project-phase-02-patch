\documentclass[10pt,a4paper]{article}

%\renewcommand{\section}{\section{\bold{#1}}

\newcommand{\blankpage}{
\newpage
\thispagestyle{empty}
\mbox{}
\newpage
}

\usepackage[colorlinks=false]{hyperref}
\hypersetup{%
  colorlinks = true,
  linkcolor  = black
}
\usepackage{geometry}
\usepackage{url}
\usepackage{booktabs}

% maths
\usepackage{bm}
\usepackage{xspace}

% algorithms
\usepackage{algorithm,algpseudocode}
\newcommand{\init}{\textbf{Init}\xspace}
\newcommand{\dokw}{\textbf{do}\xspace}
\newcommand{\upon}{\textbf{Upon}\xspace}
\newcommand{\interface}{\textbf{Interface}\xspace}
\newcommand{\crash}{\textbf{Crash}\xspace}
\newcommand{\eventname}{\textbf{EventName}\xspace}
\newcommand{\procname}{\textbf{ProcName}\xspace}
\newcommand{\timename}{\textbf{TimeName}\xspace}
\newcommand{\state}{\textbf{State}\xspace}
\newcommand{\trigger}{\textbf{Trigger}\xspace}
\newcommand{\requests}{\textbf{Requests}\xspace}
\newcommand{\indications}{\textbf{Indications}\xspace}
\newcommand{\proc}{\textbf{Procedure}\xspace}
\newcommand{\timer}{\textbf{Timer}\xspace}
\newcommand{\call}{\textbf{Call}\xspace}
\newcommand{\return}{\textbf{Return}\xspace}
\newcommand{\setup}{\textbf{Setup}\xspace}
\newcommand{\periodic}{\textbf{Periodic}\xspace}
\newcommand{\cancel}{\textbf{Cancel}\xspace}
\newcommand{\receive}[3]{\textbf{Receiver} (\textbf{#1}, \emph{sender}, #2, #3)}
\newcommand{\send}[3]{\textbf{Send} (\textbf{#1}, \emph{dest}, #2, #3)}
\algblockdefx{Interface}{EndInterface}[1]{\interface #1}{\textbf{end} \interface}
\algblockdefx{AlgState}{EndAlgState}[1]{\state #1}{\textbf{end} \state}
\algblockdefx{Requests}{EndRequests}{\requests\textbf{:}}{}
\algblockdefx{Indications}{EndIndications}{\indications\textbf{:}}{}
\algblockdefx{Upon}{EndUpon}[1]{\upon #1 \algorithmicdo}{\textbf{end} \upon}
\algblockdefx{Trigger}{EndTrigger}[1]{\trigger #1}{\textbf{end} \trigger}
\algrenewcommand\textproc{}
\algrenewcommand\algorithmicprocedure{\textbf{Procedure}}
\makeatletter
\newlength{\trianglerightwidth}
\settowidth{\trianglerightwidth}{$\triangleright$~}
\algnewcommand{\LineComment}[1]{\State \texttt{//} \textit{#1}}
\algnewcommand{\LineCommentCont}[1]{\State%
  \parbox[t]{\dimexpr\linewidth-\ALG@thistlm}{\hangindent=\trianglerightwidth \hangafter=1 \strut$\triangleright$ \emph{#1}\strut}}
\makeatother
\newcommand{\farg}[1]{\ensuremath{\textbf{arg}_{#1}}\xspace}

% % Fonts
\usepackage{times}
\usepackage[T1]{fontenc}

\def\course{Lab 01 Assignment: Algoritmos e Sistemas Distribu{\'i}dos}
\def\coursePT{Lab 01 Trabalho: Algoritmos e Sistemas Distribu{\'i}dos}

\def\titulo{\course{}}
\title{\titulo}

\def\data{\today}
\date{\data}



% Set your name here
\def\name{Alex Davidson and Nuno Preguiça}
\def\institution{
NOVA Laboratory for Computer Science and Informatics (NOVA LINCS)\\
and\\
Departamento de Inform{\'a}tica\\
Faculdade de Ci{\^e}ncias e Tecnologia\\
Universidade NOVA de Lisboa}

\author{\name\\ \institution\\ {\small (Based on notes produced by João Leitão)}}

% The following metadata will show up in the PDF properties
\hypersetup{
  colorlinks = true,
  urlcolor = black,
  pdfauthor = {\name},
  pdfkeywords = {Pseudocode, Link Abstractions}
  pdftitle = {\course: Lab 1 Problem},
  pdfsubject = {\course},
  pdfpagemode = UseNone
}

\geometry{
  body={6.5in, 9.0in},
  left=1.0in,
  top=1.0in
}

% Customize page headers
\pagestyle{myheadings}
\markright{\course - Lab1 Problem}
\thispagestyle{empty}

% Custom section fonts
\usepackage{sectsty}

% \usepackage[english,ruled]{algorithm2e}
\usepackage{amsmath}

\sectionfont{\rmfamily\mdseries\Large\textbf}
\subsectionfont{\rmfamily\mdseries\itshape\large}

% Other possible font commands include:
% \ttfamily for teletype,
% \sffamily for sans serif,
% \bfseries for bold,
% \scshape for small caps,
% \normalsize, \large, \Large, \LARGE sizes.

% Don't indent paragraphs.
\setlength\parindent{0em}

% % Make lists without bullets and compact spacing
% \renewenvironment{itemize}{
%   \begin{list}{}{
%     \setlength{\leftmargin}{1.5em}
%     \setlength{\itemsep}{0.25em}
%     \setlength{\parskip}{0pt}
%     \setlength{\parsep}{0.25em}
%   }
% }{
%   \end{list}
% }

\begin{document}

\pagenumbering{arabic} 

\maketitle

\section{Introduction: Broadcast problem}
Informally, the ``Broadcast problem'' states that a process needs to transmit the same message m to N other processes (where N is every process in the system including themselves).

We assume that the complete set of processes in the system is known a-priori: \(\pi\). We assume we have access to the Perfect Point-to-Point Link Abstraction. Finally, we assume an asynchronous system (no rounds, no failure detection).

\subsection*{Broadcast methods}
Consider the following two broadcast methods (Algorithm~\ref{alg:beb} and Algorithm~\ref{alg:rb}).

\subsubsection*{Best-Effort Broadcast}
\textit{Informal description}: Send the message to everyone, one at a time. When you receive one of these messages, just deliver it to the upper layer.

\begin{description}
  \item[BEB1 (Best-Effort validity):] For any two correct processes i and j, every message broadcasted by i is eventually delivered by j.
  \item[BEB2 (No Duplication):] No message is delivered more than once.
  \item[BEB3 (No Creation):] If a correct process j delivers a message m, then m was broadcast by some process i.  
\end{description}

\begin{algorithm}[h]
	\label{alg:beb}
	\caption{Best-Effort Broadcast pseudocode}
	\begin{algorithmic}
	\item[]
	\Interface
	  \LineCommentCont{Nothing to specify}
	\EndInterface
	
	\item[]
	\AlgState
	  \State $\pi$ \Comment{set of processes in the system}
	
		\item[]
	  \Upon{\textbf{Init}}
      \LineCommentCont{Nothing to specify}
		\EndUpon
		\item[]
	  \Upon{\textbf{bebBroadcast( $m$ )}}
      \For{$p \in \pi$}
        \State \trigger \textbf{pp2psend ( $p$, $m$ )}
	    \EndFor
		\EndUpon
		\item[]
	  \Upon{\textbf{pp2pDeliver ( $p$, $m$ )}}
	    \State \trigger \textbf{bebDeliver ( $p$, $m$ )}
		\EndUpon
	\EndAlgState
	\end{algorithmic}
\end{algorithm}

\newpage

\subsection*{Reliable Broadcast}
The Best-Effort broadcast works well until a process \emph{fails}, and thus the message is not sent to all other processes. This calls for a \emph{Reliable} broadcast, where all processes receive the necessary messages. This can be built using the Best-Effort Broadcast in Algorithm~\ref{alg:beb} as a subroutine.

\begin{description}
  \item[RB1 (Validity):] If a correct process i broadcasts message m, then i eventually delivers the message.
  \item[RB2 (No Duplications):] No message is delivered more than once.
  \item[RB3 (No Creation):] If a correct process j delivers a message m, then m was broadcast to j by some process i. 
  \item[RB4 (Agreement):] If a message m is delivered by some correct process i, them m is eventually delivered by every correct process j.
\end{description}

\begin{algorithm}[h]
	\label{alg:rb}
	\caption{Reliable Broadcast pseudocode}
	\begin{algorithmic}
	\item[]
	\Interface
	  \LineCommentCont{Nothing to specify}
	\EndInterface
	
	\item[]
	\AlgState
	  \State delivered \Comment{set of message IDs already delivered}
	
		\item[]
	  \Upon{\textbf{Init}}
      \State delivered $\leftarrow \emptyset$
		\EndUpon
		\item[]
	  \Upon{\textbf{rbBroadcast( $m$ )}}
      \State \trigger \textbf{rbDeliver($m$)}
      \State \(mid \leftarrow \textbf{generateUniqueID}(m)\)
      \State delivered \(\leftarrow\) delivered \(\cup\) \(mid\);
      \State \trigger \(\textbf{bebBroadcast}( mid, m  )\);
		\EndUpon
		\item[]
	  \Upon{\textbf{bebDeliver ( $p$, $(mid, m)$ )}}
      \If{\(mid \notin\) delivered}
        \State delivered \(\leftarrow\) delivered \(\cup\) \(mid\);
        \State \trigger \(\textbf{rbDeliver}(m)\)
        \State \trigger \(\textbf{bebBroadcast}(mid, m)\)
      \EndIf
		\EndUpon
	\EndAlgState
  \item[]
  \Procedure{\textbf{generateUniqueID}}{m}
		\LineCommentCont{Generates a unique ID for the input message \(m\).}
	\EndProcedure
	\end{algorithmic}
\end{algorithm} 

\newpage

\section{Challenge}
The Reliable Broadcast shown in Algorithm~\ref{alg:rb} consumes a lot of messages, since each process attempts to deliver all messages that it receives to every other process in the system. Asymptotically, this requires sending \(O(|\pi|^2)\) messages, which can introduce a lot of load to the network, and the processes themselves.

\subsubsection*{Exercise}
Write the pseudocode for solving the Reliable Broadcast Problem assuming the following.
\begin{itemize}
  \item A Fail stop model and a synchronous system.
  \item Your solution \textbf{must} ensure all properties of the Reliable Broadcast Problem.
  \item Your solution should ensure that in fault-free executions (i.e., when no process crashes) each process collaborates in the dissemination by sending only a \textbf{single} message.
\end{itemize}

\end{document}